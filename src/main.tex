% -*- coding: utf-8 -*-

% 算法可分页
\makeatletter
\newenvironment{breakablealgorithm}
  {% \begin{breakablealgorithm}
   \begin{center}
     \refstepcounter{algorithm}% New algorithm
     \hrule height.8pt depth0pt \kern2pt% \@fs@pre for \@fs@ruled
     \renewcommand{\caption}[2][\relax]{% Make a new \caption
       {\raggedright\textbf{\ALG@name~\thealgorithm} ##2\par}%
       \ifx\relax##1\relax % #1 is \relax
         \addcontentsline{loa}{algorithm}{\protect\numberline{\thealgorithm}##2}%
       \else % #1 is not \relax
         \addcontentsline{loa}{algorithm}{\protect\numberline{\thealgorithm}##1}%
       \fi
       \kern2pt\hrule\kern2pt
     }
  }{% \end{breakablealgorithm}
     \kern2pt\hrule\relax% \@fs@post for \@fs@ruled
   \end{center}
  }
\makeatother

\documentclass[12pt,openright]{book}

\usepackage{ifxetex}
\ifxetex
  \usepackage[bookmarksnumbered]{hyperref}
\else
  \usepackage[unicode,bookmarksnumbered]{hyperref}
\fi

\usepackage[emptydoublepage]{NKThesis}   % 中文
%\usepackage[emptydoublepage,English]{NKThesis} % 英文
\usepackage{amssymb}

%   根据需要选择 biblatex 宏包选项. 使用biblatex管理参考文献,详见百度
%\usepackage[sorting=none]{biblatex}
%\usepackage[backend=biber,style=gb7714-2015,seconds=true,gbnamefmt=lowercase,gbpub=false]{biblatex}
\usepackage{cite}
\hypersetup{colorlinks=true,
            pdfborder=0 0 1,
            citecolor=black,
            linkcolor=black}
%\usepackage{tikz}
\usepackage{amsmath}
\usepackage{algorithm}
\usepackage{algorithmic}
\usepackage{cases}
\usepackage{ulem}
\usepackage{listings} 
\usepackage{multirow}
\usepackage{graphicx}
\usepackage{caption}
\usepackage{url}
%\addbibresource{nkthesis.bib}
%\DeclareBibliographyCategory{cited}
%\AtEveryCitekey{\addtocategory{cited}{\thefield{entrykey}}}

\newtheorem{Theorem}{\hskip 2em 定理}[chapter]
\newtheorem{Lemma}[Theorem]{\hskip 2em 引理}
\newtheorem{Corollary}[Theorem]{\hskip 2em 推论}
\newtheorem{Proposition}[Theorem]{\hskip 2em 命题}
\newtheorem{Definition}[Theorem]{\hskip 2em 定义}
\newtheorem{Example}[Theorem]{\hskip 2em 例}
%\newcommand{\upcite}[1]{\textsuperscript{\textsuperscript{\cite{#1}}}}
\newcommand{\tabincell}[2]{\begin{tabular}{@{}#1@{}}#2\end{tabular}} %放在导言区 table

\begin{document}

\bibliographystyle{plain}

%  设置基本信息
%  注意:  逗号`,'是项目分隔符. 如果某一项的值出现逗号, 应放在花括号内, 如 {,}
%

\NKTsetup{%
  论文题目(中文) = 为中国之崛起而读书,
  论文题目(中文)(第二行) =知中国服务中国,
  论文题目(英文) =  Wei zhong guo zhi jue qi er du shu
  论文题目(英文)(第二行) = zhi zhongguo fuwu zhongguo,
%   论文题目(英文)(第三行) =无,
  学号           = 161099,
  姓名          = 周恩来,
  学院          = 中国,
  系别          = 服务中国,
  专业           = 服务中国,
  年级          = 1919,
  论文完成时间   = 1919年11月,
  指导教师       = 张伯苓,
  % 如果有校外导师则填写校外导师。没有填无
%   校外导师      =无,
  校外导师     = 无,
  }

% 这里包含需要到文件
% -*- coding: utf-8 -*-
% !TeX root = main.tex

% 中文摘要
\begin{zhaiyao}

\end{zhaiyao}



% 中文关键词
\begin{guanjianci}

\end{guanjianci}


% 英文摘要
\begin{abstract}


\end{abstract}


% 英文关键词
\begin{keywords}

\end{keywords} 
\tableofcontents  % 目录
% !TeX root = main.tex
% -*- coding: utf-8 -*-

\chapter{绪论}













% -*- coding: utf-8 -*-
% !TeX root = main.tex
\chapter{实验部分}


% -*- coding: utf-8 -*-
% !TeX root = main.tex
\chapter{总结与展望}









% -*- coding: utf-8 -*-
% !TeX root = main.tex
\zihaowu
\renewcommand{\bibname}{参考文献}
%\def\bibrangedash{ $\sim$ }
%\printbibliography
 \bibliography{nkthesis}
% -*- coding: utf-8 -*-

%\makeschapterhead{致谢}
\chapter*{致谢}

\include{appendices}
%% -*- coding: utf-8 -*-


\chapter*{个人简历}

\noindent 姓名:张十三\\
出生日期:1903年6月1日\\
\\
教育背景:\\  
2012年9月-2016年7月\quad 南开大学\quad 某某学院\quad\quad\quad\quad 某某学\quad\quad 学士 \\ 
2012年9月-2016年7月\quad 南开大学\quad 某某学院\quad\quad\quad\quad 某某学\quad\quad 硕士 \\
\\
硕士期间发表的学术论文:\\

\end{document}
